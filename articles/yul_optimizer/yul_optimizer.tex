\documentclass[11pt,letterpaper]{article}
\usepackage[utf8]{inputenc}
\usepackage[english]{babel}
\usepackage{latexsym,amsbsy,amssymb,amsmath,amsfonts}
\usepackage{amsthm}
\usepackage{epsfig}
\usepackage{euscript}
\usepackage{fullpage}
\usepackage{multirow,multicol,booktabs}
\usepackage[bf]{caption}
\usepackage{graphicx}
\usepackage{enumitem}
\usepackage{mathtools}
\usepackage{hyperref}
\hypersetup{
    pdfauthor={},
    pdftitle={}
    unicode,
    breaklinks,
    colorlinks=false,
    pdfborder={0 0 0}
}

\date{}

\renewcommand{\captionfont}{\small}


%%%%%%%%%%%%%%%%%%%%%%%%%%%%%%%%%%%%%%%%%%%%%%%%%%%%%%%%%%%%%
%%%%%%%%%%%%%%%%%%%%%%%%%%%%%%%%%%%%%%%%%%%%%%%%%%%%%%%%%%%%%
%%%%%%%%%%%%%%%%%%%%%%%%%%%%%%%%%%%%%%%%%%%%%%%%%%%%%%%%%%%%%


\hfuzz=0mm
\tolerance=10000
\hbadness=1000


\setlength{\parindent}{0mm}
\setlength{\parskip}{2ex plus0.5ex minus0.5ex}


\newcommand{\Sum}{\sum\limits}
\newcommand{\Prod}{\prod\limits}


\newtheorem{dummytheorem}{Dummy-Theorem}[section]
\newtheorem{definition}[dummytheorem]{Definition}
\newtheorem{lemma}[dummytheorem]{Lemma}
\newtheorem{theorem}[dummytheorem]{Theorem}
\newtheorem{proposition}[dummytheorem]{Proposition}
\newtheorem{property}[dummytheorem]{Property}
\newtheorem{corollary}[dummytheorem]{Corollary}
\newtheorem{example}[dummytheorem]{Example}
\newtheorem{remark}[dummytheorem]{Remark}
\newtheorem{fact}[dummytheorem]{Fact}
\newtheorem{claim}[dummytheorem]{Claim}
\newtheorem{subclaim}{Subclaim}[dummytheorem]
\newtheorem{conjecture}[dummytheorem]{Conjecture}

\newcommand{\uint}{\mathbb{N}}
\newcommand{\rational}{\mathbb{Q}}

\newcommand{\oli}[1]{\overline{#1}}

%%%%%%%%%%%%%%%%%%%%%%%%%%%%%%%%%%%%%%%%%%%%%%%%%%%%%%%%%%%%%%%%%%%%%%%%%%%%%%%%
%%%%%%%%%%%%%%%%%%%%%%%%%%%%%%%%%%%%%%%%%%%%%%%%%%%%%%%%%%%%%%%%%%%%%%%%%%%%%%%%
%%%%%%%%%%%%%%%%%%%%%%%%%%%%%%%%%%%%%%%%%%%%%%%%%%%%%%%%%%%%%%%%%%%%%%%%%%%%%%%%



\newcommand{\eps}{\varepsilon}
\newcommand{\card}[1]{\##1}
\newcommand{\length}[1]{\mathrm{length}({#1})}

\newcommand{\pn}[1]{\textnormal{#1}}

\newcommand{\norm}[1]{||{#1}||}
\newcommand{\Norm}[1]{\left|\left|{#1}\right|\right|}

\DeclarePairedDelimiter\ceil{\lceil}{\rceil}
\DeclarePairedDelimiter\floor{\lfloor}{\rfloor}

%%%%%%%%%%%%%%%%%%%%%%%%%%%%%%%%%%%%%%%%%%%%%%%%%%%%%%%%%%%%%%%%%%%%%%%%%%%%%%%%
%%%%%%%%%%%%%%%%%%%%%%%%%%%%%%%%%%%%%%%%%%%%%%%%%%%%%%%%%%%%%%%%%%%%%%%%%%%%%%%%
%%%%%%%%%%%%%%%%%%%%%%%%%%%%%%%%%%%%%%%%%%%%%%%%%%%%%%%%%%%%%%%%%%%%%%%%%%%%%%%%


\begin{document}
\selectlanguage{english}


\title{Optimizing Yul Code}

\author{Christian Reitwießner \\ 
{\tt chris@ethereum.org} \and etc.}


\maketitle


%\begin{abstract}
%abc
%\end{abstract}

\section{Optimizing Operations}

This section describes various operations that
can be employed to Yul code, their benefits and drawbacks.
The guiding principle for all these operations is that they
perform simple replacements that are as local as possible
in order to easy correct implementation. The second goal
is that code should always be readable in order to be
able to follow the actions of the optimizer. This is also
one reason why the Yul optimizer does not make use of
full static single assignment (SSA) form and instead
tries to keep multi-level expressions and control structures.

TODO: For each step, we should investigate which normal forms are preserved

\subsection{Disambiguator}

The disambiguator renames identifiers such that each of them has a unique name
across the whole code. This is a prerequisite for all other optimiser stages.
One of the benefits is that identifier lookup does not need to take scopes into account
and we can basically ignore the result of the analysis phase.

All subsequent stages have the property that all names stay unique. This means if
a new identifier needs to be introduced, a new unique name is generated.

The resulting code is said to be \emph{disambiguated}.

Prerequisites: None\\
Benefits: Makes other optimizer stages easier to implement\\
Drawbacks: Decreases code readability

\subsection{Function Hoister}

The function hoister moves all function definitions to the end of the topmost block. This is
a semantically equivalent transformation on disambiguated code.
This is because moving a definition to a higher-level block cannot decrease
its visibility (forward-references to functions are valid)
and variable references cannon cross function boundaries.

Prerequisites: Disambiguator\\
Benefits: Structure of the AST will be simpler (no nested functions)\\
Drawbacks: Modularity is partly lost

\subsection{Function Grouper}

The function grouper has to be applied after the disambiguator and the function hoister.
Its effect is that all topmost elements that are not function definitions are moved
into a single block which is the first statement of the root block.

After this step, a program has the following normal form:

	\verb|{ I F... }|

Where \texttt{I} is a block that does not contain any function definitions (not even recursively)
and \texttt{F} is a list of function definitions such that no function contains a function definition.

We call this normal form \emph{function grouped normal form}.
All subsequent optimizer transforms assume the code to be in this normal form.

Prerequisites: Disambiguator, Function Hoister\\
Benefits: Structure of the AST will be slightly simpler\\
Drawbacks: None

\subsection{Expression Breakup}

This step is similar to what is generally called "local value numbering", when combined
with the Expression Simplifier and the Common Subexpression Eliminator.
The idea is that new variables are introduced such that all arguments to function call
expression (including opcodes) are variables.
Outside of the statement context (i.e. for example in loop conditions)
naively applying this transform might change the order of function calls and thus
its application is limited to statements. The resulting code thus will still
have complex expressions in some cases.

Another benefit of this operation is that opcodes can be more easily reordered,
because the arguments to functions are variables, whose references can span
multiple statements.

More specifically, the transform is defined as follows:

The transform traverses the AST and performs changes to
variable declarations, assignments and expression statements.
These changes will inject new variable declarations right before
the current statement and also modify the current statement.

The breakup routine applied to an expression will inject
new variable declarations rigth before the current statement
and return the name of a new variable. In detail:

Breakup aplied to a variable just returns the variable name.

Breakup applied to a literal \verb|x| creates a new variable
\verb|v|, injects \verb|let v := x| before the current statement
and returns \verb|v|.

Breakup applied to a function call \verb|e(a1, ..., an)|
applies breakup to \verb|an|, ..., \verb|a1| (in that order),
resulting in variable names \verb|vn|, ..., \verb|v1|.
It creates a new variable \verb|v|, injects \verb|let v := e(vn, ..., v1)|
before the current statement and returns \verb|v|.

The breakup routine only affects the following statements:

Variable declaration and assignment: If it is the value is a function call,
apply breakup to its argumenst in reverse order.

Expression statement: If the expression statement is a function call,
apply breakup to its arguments in reverse order.

Note that since the recursion is applied in reverse order, the
order of function calls and opcode excution is unchanged.


Prerequisites: None\\
Benefits: Simplifies reordering of function calls\\
Drawbacks: Decreases code readability

\subsection{Pseudo-SSA Transform}

As explained in the introduction, we do not want to convert the code into a full
static single assignment form. In certain cases, it might be beneficial, though,
to at least disallow re-assignments to variables and complex expression trees.

We cannot disallow all re-assignments to variables, because of the scoping rules.
If we introduce a new variable inside a nested block, it will not be visible in the
parent block. For the assignment to \verb|a| in the following example, we
cannot introduce a new variable, because it will not be visible anymore
for the multiplication operation. In general, due to the fact that Yul
retains control structures and does not replace them by plain jumps,
it cannot use Phi functions.

\begin{verbatim}
function f(a, b) -> c {
    if eq(a, 0) {
        a := 3
    }
    c := mul(a, b)
}
\end{verbatim}

The transform is defined as follows:

We specify the transform in pseudo-code.

We traverse the AST and keep track of the current name of each local variable
in a mapping called \verb|C|.

For simplicity, we only explain single-assignments.

TODO: Reset C for function definitions?

\begin{verbatim}
transform(N, C) =
  match N
    identifier x -> C[x], C
    declaration let x := e -> let x := transform(C, e), C[x] := x
    assignment x := e ->
      allocate new variable x'
      let x' := transform(C, e), C[x] := x'
    block { S1, ..., Sn } ->
      S1', C1 := transform(S1, C)
      ...
      Sn', Cn := transform(S{n-1}, C{n-1})
      let U be a block containing a sequence of assignments of the form
        xi := Cn[xi]
      for all xi such that C[xi] != Cn[xi] and xi is not
      declared inside the block
      S1', ..., Sn', U, C
    * -> apply transform recursively
\end{verbatim}

After this transformation, a variable is guaranteed to still contain
the value of its initial declaration at a point P if TODO is there a
good condition for this?

Example transform:

\begin{verbatim}
{
    let a := 1
    {
        a := 2
    }
    let c := mul(a, 3)
}
\end{verbatim}
is transformed to
\begin{verbatim}
{
    let a := 1
    {
        let a1 := 2
        { a := a1 }
    }
    let c := mul(a, 3)
}
\end{verbatim}

Note that variables can still be re-assigned, and it is generally
advised to keep track of re-assignments, but variables retain
their values for a longer stretch of code. Because the old
values of variables are still visible, this allows the
Common Subexpression Eliminator to be more efficient.

Prerequisites: None\\
Benefits: Reduces mutability of variables\\
Drawbacks: Decreases code readability

\subsection{Functional Inliner}

The functional inliner depends on the disambiguator, the function hoister and function grouper.
It performs function inlining such that the result of the inlining is an expression. This can
only be done if the body of the function to be inlined has the form \verb|{ r := E }| where
\verb|r|
is the single return value of the function, \verb|E| is an expression and all arguments in the
function call are so-called movable expressions. A movable expression is either a literal, a
variable or a function call (or EVM opcode) which does not have side-effects and also does not
depend on any side-effects.

As an example, neither \verb|mload| nor \verb|mstore| would be allowed.

This inliner is useful for expressions where Expression Breakup cannot be applied.

\subsection{Full Function Inliner}

This inliner can inline any function call, but has the drawback that the resulting
code will contain more statements. It is best applied after Expression Breakup.

Statements of the form \verb|let x1, ..., xn := f(a1, ..., an)|, where \verb|f|
is a function defined as \verb|function f(v1, ..., vn) -> r1, ..., rn B|
are replaced as follows:

Let \verb|B'| be a copy of \verb|B| where each variable \verb|v| is replaced by
a new unique name \verb|C[v]| and \verb|C[ri] = xi|.
Then the statement is replaced by the following statements:
\begin{verbatim}
let x1, ..., xn
let C[v1] := a1
...
let C[vn] := an
B
\end{verbatim}

Note that \verb|a1|, ..., \verb|an| must be variables.

Prerequisites: Expression Breakup

\subsection{Rematerialisation}

The rematerialisation stage tries to replace variable references by the expression that
was last assigned to the variable. This is of course only beneficial if this expression
is comparatively cheap to evaluate. Furthermore, it is only semantically equivalent if
the value of the expression did not change between the point of assignment and the
point of use. The main benefit of this stage is that it can save stack slots if it
leads to a variable being eliminated completely (see below), but it can also
save a DUP opcode on the EVM if the expression is very cheap.

The algorithm only allows movable expressions (see above for a definition) in this case.
Expressions that contain other variables are also disallowed if one of those variables
have been assigned to in the meantime. This is also not applied to variables where
assignment and use span across loops and conditionals.

TODO: Rewrite this assuming pseudo-SSA and especially Expression Breakup.
After Expression Breakup, movability can be relaxed to something like
"the expression commutes with all expressions in between".

\subsection{Unused Definition Pruner}

If a variable or function is not referenced, it is removed from the code.
If there are two assignments to a variable where the first one is a movable expression
and the variable is not used between the two assignments (and the second is not inside
a loop or conditional, the first one is not inside), the first assignment is removed.


\subsection{Function Unifier}

\subsection{Expression Simplifier}

This step can only be applied for the EVM-flavoured dialect of Yul. It applies
simple rules like \verb|x + 0 == x| to simplify expressions.

\subsection{Ineffective Statement Remover}

This step removes statements that have no side-effects.

\subsection{Common Subexpression Eliminator}

\subsection{Redundant Writes Eliminator}

\subsection{Write Delay}

This stage swaps two commuting statements with the goal of delaying writes to
memory or storage. The goal is to move writes to the same location together,
so that one of them can be removed by the Redundant Writes Eliminator.


\section{Order of Operations}

Now we want to discuss a reasonable order in which
the operations mentioned above should be applied.
Since some operations are more or less opposite to each other,
the order has a big influence on the resulting code.

\end{document}
    
